\documentclass[a4paper,twoside]{article}
\usepackage[small]{caption}
\usepackage{epsfig}
%\usepackage{subfigure}
\usepackage[subrefformat=parens,labelformat=parens]{subfig}
\usepackage{graphicx}
\usepackage{calc}
\usepackage{amssymb}
\usepackage{amstext}
\usepackage{amsmath}
\usepackage{amsthm}
\usepackage{multicol}
\usepackage{pslatex}
\usepackage{apalike}
\usepackage{SCITEPRESS}
\usepackage{mathrsfs}
\captionsetup[subfloat]{farskip=0pt,nearskip=0pt,captionskip=5pt}


%\subfigtopskip=0pt
%\subfigcapskip=0pt
%\subfigbottomskip=0pt

\begin{document}

\title{Homotopy Surface Cutting Using Edges' Sources in Geodesic Distance}

\author{\authorname{Anuwat Dechvijankit,Author2 and Author3}
\affiliation{Department of Computational Intelligence and Systems Science, Tokyo Institute of Technology, Japan}
\email{dechvijankit.a.aa@m.titech.ac.jp}
}

\keywords{geodesic distance, graph cut, homotopy, surface parameterization}


\abstract{Topology is a property in surfaces that plays a major role in computer graphics. Processing or analysis between two surfaces generally require both of them to be same topology. There are many tools or applications that require disk topology surfaces as input such as parameterization or remeshing. Therefore, we need convert any surfaces to be same as topological disk. The common procedure is to define edges graph inside surface that be split into two edges and turns surface into topological disk. We call it as homotopy cutting. Problems become more difficult when dealing with high genus surfaces such as torus. Based on a novel method, we present an enhancement method to define cut graph in high-genus surface for homotopy cutting. By using geodesic properties of each edge, we can generate equally or more suitable edges graph than original method while remains performance and stability as original method.}

\onecolumn \maketitle \normalsize \vfill

\section{\uppercase{Introduction}}
\label{sec:introduction}

\noindent Geometry processing is an important research in 3D computer graphics field. Without efficient algorithms, it is very difficult to develop any kinds of advance application for end-users. Some of important applications in 3D computer graphics, such as texture mapping \cite{Bennis:1991:PSF:127719.122744}, normal mapping \cite{Cohen:1998:AS:280814.280832}, remeshing \cite{Hormann00quadrilateralremeshing} and parameterization \cite{Tutte:1963,Floater:1997:PSA:248299.248308} require specific topology of input mesh. There are many cases that topological disk surface is specified for further processing. With such requirement of topology in input mesh, it has impact on several researches in computer graphics and graph theory. There are many properties in each mesh such as closed/open, holes and genus.

When dealing with mesh that require disk topology input, there is different measure on each kinds of meshes. Open surface has same topology as disk already which can pass directly but may need to take care in case of containing holes. The problems arise when dealing with closed surface since it has different topology from disk. The process to cutting surface into the disk is required. In case of sphere topology, it does not require much processes; only short graph edge is necessary. However, there is some processes to ensure quality that require more graph edges in homotopy cutting. The problem becomes more complex and more interesting when dealing with high genus surfaces. 
   
This present paper solves homotopy cutting on high genus surfaces. Our approach is an enhancement of a novel method \cite{Gu:2002:GI:566654.566589} in homotopy cutting; cutting surface into disk. A benefit of this method is to able to handle any kinds of 2-manifold surfaces, regardless from topology specific. We presents an algorithms that create cut graph on the area where geodesic path came from difference directions in exact geodesic distance \cite{Mitchell:1987:DGP:33367.33372,Surazhsky:2005:FEA:1073204.1073228} (see example in figure \ref{fig:geodesic rocket arm}). With an few extra calculation,  we can define equally or more appropriate cut graph from original method while remains performance and stability.
\begin{figure}[!h]
	%\vspace{-0.2cm}
	\centering
	{\includegraphics[width=0.9\columnwidth]{images/geodesic_rocket-arm.png}}
	\caption{Geodesic distance radius from a starting point on genus 1 rocket arm model. At the hole, we can see some sharp pattern which can be recognize as geodesic path came from different directions.}
	\label{fig:geodesic rocket arm}
\end{figure}

\subsection{Notations}
Before we explains various algorithms about homotopy including ours, let us define basic notations. We represent 2-manifold triangular surface or mesh $\mathscr{M}:=(V,F)$ where $V:=\{ v_{i}\in \mathbb{R}^3 : i = 1, ... , n_v\}$ is a set of $n_v$ vertices and $F:=\{ f_{i}(a,b,c) : a,b,c = 1, ... , n_v : a \neq b \neq c\}$ is a set of $n_f$ faces information as index of vertices. Let define $E:=\{ e_{i}(a,b) : a,b = 1, ... , n_v : a \neq b\}$ as a set of $n_e$ edges information that be found in surface $\mathscr{M}$. Mesh has genus $g$ topology. 


\section{\uppercase{Related Works}}
\label{sec:related works}
\noindent From previous researches in topological converting topic in the past few years, there was a novel study the problem of cutting a topological surface into the disk efficiently by \cite{Erickson:2002:OCS:513400.513430}. They have proposed a cutting method which has some elegant theoretical guarantees but is complex to implement. It finds the shortest loop path connecting a vertex to the vertex itself by using a front propagation technique, and then tests to see if the considering loop path reduces the surface genus or simply cut the surface into two pieces. It has topologically-sufficient cut as $2g$ loops. The generation of minimal length cuts that convert a high genus surface into a topological disk is a NP-hard problem. The method is a brute force approach which consumes a lot of time. it has approximation of the shortest cut graph in $O(g^2 n \log n)$ where $n$ donates complexity of the surface. In more study from \cite{Erickson:2005:GOH:1070432.1070581}, they studied about greedy homotopy basis and improvement speed in $O(n \log n)$
by using a straightforward application of Dijkstra's shortest path algorithm \cite{Dijkstra59anote}. 

One of important in efficiency is to compute non-trivial cycles on orientable surfaces. Non-trivial cycles mean non-contractible and non-separating cycles which guarantees cutting topological surface into disk. Recently, \cite{Kutz:2006:CSN:1137856.1137919} presents an algorithm that computes a shortest non-trivial cycles on orientable combinatorial surface of bounded genus in  $O(n \log n)$. The algorithm is based on universal-cover constructions to find short cycles.

There are studies that trying to define cut graph by properties of surfaces. Study from \cite{Patane:2007:FCB:1224804.1224947} presents an algorithm that build up the cut graph on the iso-contours from Reeb graph which codes in a combinatorial structure the topology of a given surface $\mathscr{M}$ and connect loops together. Another study from \cite{Jin:2013:CSH:2396897.2396971}, presents algorithm to compute the shortest homotopic loop with negative Euler characteristic based on the surface hyperbolic uniformization metric. They also demonstrate two applications: constructing extremal quasi-conformal mappings between same topology surfaces, and detecting homotopy between two paths or cycles on a surface. 

There is a iterative method called "geometry images" by \cite{Gu:2002:GI:566654.566589}. This paper presents a remeshing approach using square surface parameterization to create  using mapping between disk topology irregular surface $\mathscr{M}$ in $\mathbb{R}^3$ domain and square planar in $\mathbb{R}^2$ domain and sample positions to get a regular positions. To get low error on remeshing, they present how to create cut graph from any kinds of surface $\mathscr{M}$ regardless from pre-analysis such as topology and boundary edges.

\begin{figure}[!h]
	%\vspace{-0.2cm}
	\centering
	\subfloat[Geometry of surface]{\label{fig:gim3d}\includegraphics[width=0.45\columnwidth]{images/gim_bunny3D.png}}
	\subfloat[Geometry image]{\label{fig:gim2d}\includegraphics[width=0.45\columnwidth]{images/gim_bunny2D.png}}
	
	\caption{A geometry image.}
	\label{fig:gim figure}
\end{figure}

Since our approach is based on geometry images, we explain how it creates cut graph for homotopy cutting on irregular surface $\mathscr{M}$ with genus $g$ property in section \ref{sec:previous algorithm}.
\section{\uppercase{Previous Algorithm}}
\label{sec:previous algorithm}
\noindent \cite{Gu:2002:GI:566654.566589} algorithm is divided into two parts, homotopy cutting and its augmentation. The augmentation aims to improve subsequent square planar domain parameterization. We explains the first part that involve defining cut graph and convert surface $\mathscr{M}$ into disk.



\begin{figure*}[t]
	\centering		
	\subfloat[\label{fig:OriginalGenusReduceMethodStepByStep-a}]{\includegraphics[width=0.400\columnwidth]{images/fig-original_genus_reduce_method_step_by_step-a.png}}\hspace{10pt}
	\hspace{0.000\columnwidth}
	\subfloat[\label{fig:OriginalGenusReduceMethodStepByStep-b}]{\includegraphics[width=0.400\columnwidth]{images/fig-original_genus_reduce_method_step_by_step-b.png}}\hspace{10pt}		
	\hspace{0.000\columnwidth}
	\subfloat[\label{fig:OriginalGenusReduceMethodStepByStep-c}]{\includegraphics[width=0.400\columnwidth]{images/fig-original_genus_reduce_method_step_by_step-c.png}}\hspace{10pt}		
	\hspace{0.000\columnwidth}
	\subfloat[\label{fig:OriginalGenusReduceMethodStepByStep-d}]{\includegraphics[width=0.400\columnwidth]{images/fig-original_genus_reduce_method_step_by_step-d.png}}		
	\caption{Processes after removing a seed triangle from mesh. Gray areas mean that there are still triangle, while white areas mean triangles have been removed. Dash lines mean edges that are adjacent to only one triangle at the moment. \subref{fig:OriginalGenusReduceMethodStepByStep-a} shows the moment after removing the seed triangle; that is, edges e1,e2 and e3 are only edges that adjacent only one triangle. Assume that edge e1 has the smallest geodesic distance from the seed triangle at the moment. \subref{fig:OriginalGenusReduceMethodStepByStep-b} shows the result of removing edge e1 and face F1 from the condition. At this moment, edge e4 is also adjacent to only one triangle same as edge e2. Let e2 have smaller geodesic distance than edges e3 and e4.  \subref{fig:OriginalGenusReduceMethodStepByStep-c} shows the result of removing edge e2 and face F2 without removing edge e4. Because of the edge e4 is not adjacent any triangle after removing face F2, it is excluded from adjacent only one triangle's condition. The edge e4 becomes a candidate of seam-cutting edge. \subref{fig:OriginalGenusReduceMethodStepByStep-d} shows the result of next step from \subref{fig:OriginalGenusReduceMethodStepByStep-c} that removing triangle keep spread from the seed triangle.}
	\label{fig:OriginalGenusReduceMethodStepByStep}
\end{figure*}

At the beginning in the method , if the mesh has boundaries, let $\mathscr{B}$  be the set of original boundary edges that remain unchanged in the whole process and will be included in final cut graph ${\rho}$. It first starts by removing a single seed triangle from the mesh. At this moment, each edge of the seed triangle is adjacent to only one triangle respectively (see figure \subref*{fig:OriginalGenusReduceMethodStepByStep-a}).  After removing the seed triangle from the mesh, there are two phases of processing.

In the first phase, it repeatedly detect an edge adjacent exactly to one triangle that is not in $\mathscr{B}$, and remove both the edge and the triangle from the mesh structure. The rest two edges are left (see figure \subref*{fig:OriginalGenusReduceMethodStepByStep-b}). If the rest edges of the removing triangle are not adjacent any triangle, then the edges will become one of candidate cut graph (see figure \subref*{fig:OriginalGenusReduceMethodStepByStep-c}). Generally, removing one edge and one triangle triggers more two edges to be adjacent to  only one triangle further. Considering from mentioned condition, the early detected edges have to be the ones of the seed triangle so the removal of edges and triangles will keep spreading out from the seed triangle according to geodesic distance in order to get minimum radius result (see figure \subref*{fig:OriginalGenusReduceMethodStepByStep-d}). Since a 2-manifold triangle mesh is being processed, any triangle can be accessed from other triangles with some paths. Because of that, any triangle will be removed eventually.Therefore, this phase ends when there is no triangle left and there remain only edges and their vertices (see figure \subref*{fig:fig-original_genus_reducing_process-c}) as candidate cut graph edges. At this point, the cut $\rho$ consists of a set of connecting $2g$ loops.
In the second phase, we again iteratively detect a valence-1 vertex and its corresponding edge, and remove both the vertex and the edge (see figure \ref{fig:fig-original_remove_dangling_edges_step_by_step}\subref{fig:fig-original_remove_dangling_edges_step_by_step-a} to ~\subref{fig:fig-original_remove_dangling_edges_step_by_step-e}). The purpose of this phase is to remove unnecessary dangling edges in the first phase. The dangling edges will be repeatedly trimmed away until there is no valence-1 vertex in the cut $\rho$ left. There are only edges that form connected loops as cut-paths in the cut $\rho$ (see figure \ref{fig:fig-original_remove_dangling_edges_step_by_step-f}). At Last, all cut graph loops in $\rho$ are straightened by computing a local shortest path in each loop. Finally, the connected $2g$ loop cut graph in $\rho$ is homotopy basis: convert surface into a topological disk patch. See figure \ref{fig:fig-original_genus_reducing_process} for overall process of finding graph cut on a genus-3 mesh.


\begin{figure}[bh!]
	\centering		
	\subfloat[\label{fig:fig-original_remove_dangling_edges_step_by_step-a}]{\includegraphics[width=0.40\columnwidth]{images/fig-original_remove_dangling_edges_step_by_step-a.png}} \hspace{10pt}
	\subfloat[\label{fig:fig-original_remove_dangling_edges_step_by_step-b}]{\includegraphics[width=0.40\columnwidth]{images/fig-original_remove_dangling_edges_step_by_step-b.png}}\\		
	\subfloat[\label{fig:fig-original_remove_dangling_edges_step_by_step-c}]{\includegraphics[width=0.40\columnwidth]{images/fig-original_remove_dangling_edges_step_by_step-c.png}}	\hspace{10pt}	
	\subfloat[\label{fig:fig-original_remove_dangling_edges_step_by_step-d}]{\includegraphics[width=0.40\columnwidth]{images/fig-original_remove_dangling_edges_step_by_step-d.png}}	\\
	\subfloat[\label{fig:fig-original_remove_dangling_edges_step_by_step-e}]{\includegraphics[width=0.40\columnwidth]{images/fig-original_remove_dangling_edges_step_by_step-e.png}}	\hspace{10pt}
	\subfloat[\label{fig:fig-original_remove_dangling_edges_step_by_step-f}]{\includegraphics[width=0.40\columnwidth]{images/fig-original_remove_dangling_edges_step_by_step-f.png}}	 
	\caption[]{Process on removing dangling edges. Numbers on vertices indicate present valence number. We focus on the removing of blue dangling edges. \subref{fig:fig-original_remove_dangling_edges_step_by_step-a} shows initial state where there are two valence-1 edges in blue dangling edges at the moment. \subref{fig:fig-original_remove_dangling_edges_step_by_step-b} shows the next step from \subref{fig:fig-original_remove_dangling_edges_step_by_step-a} that removed one of valence-1 edge along with its vertex. The order of removing is not important. \subref{fig:fig-original_remove_dangling_edges_step_by_step-c} and \subref{fig:fig-original_remove_dangling_edges_step_by_step-d} show iteration process of removing valence-1 edge and vertex on blue dangling edges. \subref{fig:fig-original_remove_dangling_edges_step_by_step-e} shows that all blue dangling edges have been removed. \subref{fig:fig-original_remove_dangling_edges_step_by_step-f} shows the process of removing other dangling edges until valence-1 edge has not been found.}
	\label{fig:fig-original_remove_dangling_edges_step_by_step}
\end{figure}
For the case of closed surface of genus 0, the overall processes from this part will generate the cut $\rho$ that consists of only one vertex. To be able to map into planar domain, we add two adjacent edges of the vertex into the cut graph $\rho$. On the other hand, for the case of a mesh having one or more holes, it will result in connected graphs between any two holes.
\begin{figure}[h!]
	\centering		
	\subfloat[\label{fig:fig-original_genus_reducing_process-a}]{\includegraphics[width=0.31\columnwidth]{images/fig-original_genus_reducing_process-a.png}}
	\hspace{0.00\columnwidth}
	\subfloat[\label{fig:fig-original_genus_reducing_process-b}]{\includegraphics[width=0.31\columnwidth]{images/fig-original_genus_reducing_process-b.png}}
	\hspace{0.00\columnwidth}
	\subfloat[\label{fig:fig-original_genus_reducing_process-c}]{\includegraphics[width=0.31\columnwidth]{images/fig-original_genus_reducing_process-c.png}}
	\hspace{0.00\columnwidth}
	\subfloat[\label{fig:fig-original_genus_reducing_process-d}]{\includegraphics[width=0.31\columnwidth]{images/fig-original_genus_reducing_process-d.png}}
	\hspace{0.00\columnwidth}
	\subfloat[\label{fig:fig-original_genus_reducing_process-e}]{\includegraphics[width=0.31\columnwidth]{images/fig-original_genus_reducing_process-e.png}}
	\hspace{0.00\columnwidth}
	\subfloat[\label{fig:fig-original_genus_reducing_process-f}]{\includegraphics[width=0.31\columnwidth]{images/fig-original_genus_reducing_process-f.png}}
	\caption[]{Process of genus-reduce cutting by sequentially removing triangles, edges and vertices. \subref{fig:fig-original_genus_reducing_process-a} shows a closed genus-3 mesh with one seed triangle (yellow dot). \subref{fig:fig-original_genus_reducing_process-b} shows the intermediate state of mesh when removing triangles and edges that are adjacent to only one triangle. \subref{fig:fig-original_genus_reducing_process-c} shows edge skeleton mesh after removing all triangles. \subref{fig:fig-original_genus_reducing_process-d} and \subref{fig:fig-original_genus_reducing_process-e} show the intermediate state of mesh when removing edges and vertices that are adjacent to only one edge. \subref{fig:fig-original_genus_reducing_process-f} shows final result of straightened cutting edges after removing edges and vertices.}
	\label{fig:fig-original_genus_reducing_process}
\end{figure}

\section{\uppercase{Geodesic Distance}}
\label{sec:geodesic distance}
\noindent Since \cite{Gu:2002:GI:566654.566589} algorithm creates front propagation on geodesic distance. Without specific algorithm, we consider it as exact geodesic distance, proposed by \cite{Mitchell:1987:DGP:33367.33372} as know as MMP algorithm. It computes exact shortest paths on a triangular mesh. The paths typically cut cross faces in the mesh, difference from typical Dijkstra shortest paths \cite{Dijkstra59anote} that run cross edges in the mesh.

\begin{figure}[!h]
	%\vspace{-0.2cm}
	\centering
	{\includegraphics[width=0.75\columnwidth]{images/mmp_algorithm.png}}
	\caption{Propagation scheme of MMP algorithm: interval $i_{\alpha}$ on edge $e_a$ propagate distance pencil paths to adjacent edges $e_b$ and $e_c$ }
	\label{fig:mmp algorithm}
\end{figure}

MMP algorithm create geodesic path for "single source , all destination" scheme. The algorithm computes a set of intervals of each edge. An interval represents accessible pencil of lines from its pseudo-source. Each internal also acts as pseudo-source to propagate across faces of the rest of mesh. The algorithm propagates distance information out from source in a Dijkstra-like fashion which can traceback any positions on mesh back to the source.

The performance of MMP algorithm : They prove a worst case
running time of $O(n^2 \log n)$ when $n$ is number of mesh edges. However in practical calculation, it can achieve on 100K triangles mesh within a few seconds. Also there is approximate version of MMP algorithm proposed by \cite{Surazhsky:2005:FEA:1073204.1073228} that can speed up calculation by trying to merge
interval with an adjacent intervals on the same edge before propagation.

\section{\uppercase{Our approach}}
\label{sec:result}

\subsection{Edges That Having Difference Pseudo-Source Intervals}

\subsection{Edges That Intervals Fails to Propagate}

\subsection{Priority Removing-Edge Inside Queue }

\section{\uppercase{Results}}
\label{sec:result}
\section{\uppercase{Conclusions}}
\label{sec:conclusion}

\noindent conclusion....



\section*{\uppercase{Acknowledgements}}
\noindent The images in figure \ref{fig:gim figure} and \ref{fig:fig-original_genus_reducing_process}  are from \cite{Gu:2002:GI:566654.566589} paper and presentation file.


\vfill
\bibliographystyle{apalike}
{\small
\bibliography{genus_cutting}}



\vfill
\end{document}

